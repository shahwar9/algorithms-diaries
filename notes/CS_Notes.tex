\documentclass[12pt]{article}
\usepackage[english]{babel}
\usepackage{natbib}
\usepackage{url}
\usepackage[utf8x]{inputenc}
\usepackage{amsmath}
\usepackage{graphicx}
\graphicspath{{images/}}
%\usepackage{parskip}
\usepackage{fancyhdr}
\usepackage{vmargin}
\usepackage{float}
\usepackage{caption}
\usepackage{subcaption}
\usepackage{pythonhighlight} 
\setmarginsrb{2 cm}{2.5 cm}{2 cm}{1.5 cm}{1 cm}{1.5 cm}{1 cm}{1.5 cm}
\renewcommand{\vec}[1]{\mathbf{#1}}
\usepackage{xcolor}
\usepackage{soul}
\newcommand{\mathcolorbox}[2]{\colorbox{#1}{$\displaystyle #2$}}
\title{CS Fundamentals}                                % Title
\author{Shahwar Saleem}                               % Author
\date{12 Sept 2015}                                         % Date

\makeatletter
\let\thetitle\@title
\let\theauthor\@author
\let\thedate\@date
\makeatother

\pagestyle{fancy}
\fancyhf{}
\rhead{\theauthor}
\lhead{\thetitle}
\cfoot{\thepage}

\begin{document}

%%%%%%%%%%%%%%%%%%%%%%%%%%%%%%%%%%%%%%%%%%%%%%%%%%%%%%%%%%%%%%%%%%%%%%%%%%%%%%%%%%%%%%%%%

\begin{titlepage}
    \centering
    \vspace*{0.5 cm}
    \includegraphics[scale = 0.75]{logo_waterloo.png}\\[1.0 cm]  % University Logo
    %\textsc{\LARGE CS Fundamentals }\\[1.0 cm] 
    % \textsc{\Large ECE-657}\\[0.5 cm]               % Course Code
    \rule{\linewidth}{0.2 mm} \\[0.4 cm]
    { \huge \bfseries \thetitle}\\
    \rule{\linewidth}{0.2 mm} \\[1.5 cm]
    

    
    
    
    
    
    
    
\end{titlepage}

%%%%%%%%%%%%%%%%%%%%%%%%%%%%%%%%%%%%%%%%%%%%%%%%%%%%%%%%%%%%%%%%%%%%%%%%%%%%%%%%%%%%%%%%%

\tableofcontents
\pagebreak

%%%%%%%%%%%%%%%%%%%%%%%%%%%%%%%%%%%%%%%%%%%%%%%%%%%%%%%%%%%%%%%%%%%%%%%%%%%%%%%%%%%%%%%%%

\section{Dynamic Programming}

\begin{itemize}
\item Dynamic programming is a technique to solve a subset of problems in an efficient way. 

\item There are problems which take exponential time to solve if DP is not used. 

\item DP is enhanced recursion. 

\item DP is an improved approach based on recursive solution. DP is recursion with Storage. 

\item Recursion might calculate a same value over an over again within its nested calls. DP uses a storage mechanism to avoid making inefficient recursive calls. 

\item Recursion + Table = Memoization

\item Top-Down approach is when we only use Storage of table, to intelligently calculate further sub-problems. 
\end{itemize}

\subsection{How to check a problem for DP?}
\begin{enumerate}
\item There is a choice to add or remove an item.
\item An optimal solution is required.  
\end{enumerate}

\subsection{Approach new DP Problem}
\begin{itemize}
\item Write a recursive solution to the problem, EASY.
\item Memoize the recursive function. 
\item Then go for TOP-DOWN approach. 
\end{itemize}

\pagebreak

\subsection{Problem Pattern}
There are following 10 problem patterns which apply DP:

\begin{enumerate}
\item 0-1 Knapsack
	\begin{itemize}
	\item Subset sum
	\item Equal sum partition
	\item Count of subset 
	\item Minimum subset sum difference
	\item Count the number of subset with a given difference
	\item Target sum
	\end{itemize}
\item Unbounded Knapsack
\item Fibonacci
\item LCS: Longest Common Subsequence
\item LIS: Longest Increasing Subsequence
\item Kadane\'s Algorithm
\item Matrix Chain Multiplication*
\item DP on Trees
\item DP on Grid
\item Others

\end{enumerate}

\subsection{Detailed Knapsack Visit}

\subsubsection*{Problem Statelemt}
Knapsack problem is when we are given a bag of weight W, a weight array Wt and a value array Val.

We have to output the maximum value output with choices of input in bag. The condition is sum of weights is less than or equal to W.

Knapsack has following types:

\begin{itemize}
\item 0-1 Knapsack: value is added or not.
\item Fractional Knapsack: we can add fraction of an item.
\item Unbounded Knapsack: Item can be repeated!
\end{itemize}

\subsubsection{Knapsack: Recursive}

\begin{python}
def recursive_knapsack(wt, val, W):

    # Smallest possible input
    if len(wt) == 0 or W == 0:
        return 0

    # Get rid of last item because weight of item is greater
    # than the W itself.
    if wt[-1] > W:
        # Advance in recursion without last item.
        return recursive_knapsack(wt[:-1], val[:-1], W)
    else:
        # Weight of last item is within W, thus, 2 possibilities are there:
        # max(consider item, not consider item)
        return max(val[-1] + recursive_knapsack(wt[:-1], val[:-1], W-wt[-1]),
                   recursive_knapsack(wt[:-1], val[:-1], W))
\end{python}

\begin{itemize}
\item Method calls itself with a smaller sub problem.
\item Repetitive calls to same method utilizes call stack. 
\item Recursive solutions are simple but inefficient as system stacks may overflow. 
\end{itemize}

Whenever there is a single recursive call, DP might not be best to use. DP is helpful when 2 or more recursive calls are present within the same method.

To come up with recursive solution:
\begin{itemize}
\item Think of the smallest possible input in the problem (BASE CONDITION)
\item Think about choice diagram. If there is an item, make a small condition diagram of effects or including or not including that item.

\end{itemize}

\pagebreak
\subsubsection{Knapsack: Memoize}

\begin{python}
def knapsack_memioze(wt, val, W):
    t = []

    n = len(wt)

    # Memoization table.
    for i in range(n + 1):
        t.append([-1] * (W + 1))

    def helper_memoize(wt, val, W, n):

        # Smallest possible input
        if n == 0 or W == 0:
            return 0

        if t[n][W] != -1:
            # If memory table already has that entry calculated, use it.
            return t[n][W]

        # Get rid of last item because weight of item is greater
        # than the W itself.
        if wt[n - 1] > W:
            # Advance in recursion without last item.
            # but store the results in the memory table.
            t[n][W] = helper_memoize(wt, val, W, n - 1)
            return t[n][W]
        else:
            # Weight of last item is within W, thus, 2 possibilities are there:
            # max(consider item, not consider item)
            # store the results in memory table.
            t[n][W] = max(val[n - 1] + helper_memoize(wt, val, W - wt[n - 1], n - 1),
                          helper_memoize(wt, val, W, n - 1))

            return t[n][W]

    return helper_memoize(wt, val, W, n)
\end{python}

\begin{itemize}
\item Memoization is a combination of recursive solution and storage. 

\item Some extra storage is used to avoid extra recursive calls. This storage stores values that a recursive solution tends to calculate over and over again. 

\item Recursion is still involves, therefore, there is a chance for the stack to overflow. 

\item Therefore, memiozation is useful where number of recursive calls are less than system stack.

\end{itemize}
\pagebreak
\subsubsection{Knapsack: Top Down DP}

\begin{python}
    n = len(wt)

    # DP One line initialization.
    t = [[0 if i == 0 or j == 0 else -1 for j in range(W+1)] for i in range(n+1)]

    # Because initialization is already done for n = 0, W = 0.
    for i in range(1, n+1):
        for j in range(1, W+1):
            if wt[i-1] <= j:
                # Convert n to i and W to j from recursive solution.
                t[i][j] = max(val[i-1]+t[i-1][j-wt[i-1]], t[i-1][j])
            else:
                t[i][j] = t[i - 1][j]

            pprint(t)

    return t[n][W]
    
# The final DP table becomes:
[[0, 0, 0, 0, 0, 0, 0, 0, 0, 0, 0],
 [0, 3, 3, 3, 3, 3, 3, 3, 3, 3, 3],
 [0, 3, 3, 3, 4, 7, 7, 7, 7, 7, 7],
 [0, 3, 3, 3, 4, 7, 8, 8, 8, 9, 12],
 [0, 3, 3, 3, 4, 7, 8, 8, 10, 10, 12]]
\end{python}

\subsection{Problems}

\subsubsection{Subset Sum Problem}

Input: [ 2  3  7  8  10 ]

Sum: 11

Is there a subset of array which has sum 11?

Output: True/False

\textbf{Similarity}: Sum is Weight of Knapsack. If there is only 1 array consider it Weight array in Knapsack. There is also choice weather to add an element in subset or not. 


\end{document}
 